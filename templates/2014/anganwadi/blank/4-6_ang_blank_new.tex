\documentclass[12pt]{article}
%%\usepackage[normalmargins]{savetrees}
\usepackage[a3paper,top=1cm,left=0.3cm,right=0.5cm,bottom=3.3cm,landscape]{geometry}
\usepackage{rotating,makecell}
\usepackage{fontenc}
\usepackage{draftwatermark}
%% \usepackage{xunicode}
\usepackage{xltxtra}
\usepackage{graphicx}
\usepackage{lastpage}
\usepackage{fp}
\usepackage{fancyhdr}
\pagestyle{fancy}

\def\heading{2013-14 Anganwadi assessment}
%% Headers and footers
\lhead{\includegraphics[scale=0.12]{/home/brijesh/form_generation/logo.jpg}}
\chead{\Large{4 - 6 Years age group}\\ID: \rule{2in}{0.5pt} Name: \rule{2in}{0.5pt}  Type: \rule{2in}{0.5pt}}
\rhead{\Large{Project: \rule{2in}{0.5pt} Circle: \rule{2in}{0.5pt}\\ Worker: \rule{2in}{0.5pt}}}
\cfoot{\small{Research and Evaluation -- Akshara -- Anganwadi Assessment -- August-2014  ©Akshara Foundation}}
\rfoot{\small{\thepage of \pageref{LastPage}}}


%\nomakegapedcells
%\renewcommand\theadset{\renewcommand\arraytretch{1}%
%    \setlength\extrarowheight{0cm}}
%\renewcommand\theadgape{0cm}
%\renewcommand\cellgape{}
%\setcellgapes[b]{-0.5cm}

\setlength{\headheight}{2.2cm}
\renewcommand{\headrulewidth}{0.4pt}
\renewcommand{\footrulewidth}{0.4pt}

\newcommand{\qsetnum}{%
\FPsub{\qnum}{\thepage}{1}
\FPdiv{\qnum}{\thepage}{4}%
\FPtrunc{\tqnum}{\qnum}{0}%
\FPsub{\qnum}{\qnum}{\tqnum}%
\FPmul{\qnum}{\qnum}{4}%
\FPtrunc{\qnum}{\qnum}{0}%
\FPifeq{\qnum}{0}
   \FPset{\qnum}{4}%
\fi
\FPprint{\qnum}
}%

\newcommand{\kn}[1]{%Kannada text
{\fontspec[Script=Kannada]{Kedage}%
#1
}}

%\newcommand{\grading}[1]{\begin{sideways}\textbf{#1}\end{sideways}}
%\newcommand{\instruct}[1]{\begin{sideways}\textbf{#1}\end{sideways}}
\newcommand{\question}[1]{\begin{sideways}#1\end{sideways}}
\renewcommand{\arraystretch}{1.4}

\title{\heading}
\begin{document}

%\fontspec[Script=Kannada]{Kedage}
\SetWatermarkText{Worker Copy}

\begin{tabular}{|p{2cm}|p{3cm}|p{2.8cm}|p{2.8cm}|p{1cm}|p{1cm}|l|l|l|l|l|l|l|l|l|l|l|l|l|l|l|l|l|l|l|l|l|l|l|l|}
\hline

& & & & & & & & \multicolumn{8}{|c|}{\kn{ಸಾಮಾನ್ಯ ಅರಿವು}} 
& \multicolumn{5}{|c|}{\kn{ಸೂಕ್ಷ್ಮ ಚಲನೆ}}
& \multicolumn{9}{|c|}{\kn{ಶಿಕ್ಷಣ ಪೂರ್ವ ಬರೆಯುವುದು}}
\\ \hline

%\multicolumn{8}{|c|}{\Large{\makecell[b]{Project: \rule{2in}{0.5pt} \\ \\ Circle: \rule{2in}{0.5pt} \\ \\ ID: \rule{2in}{0.5pt} \\  \\ Name: \rule{2in}{0.5pt} \\  \\ Type: \rule{2in}{0.5pt} \\  \\ Worker: \rule{2in}{0.5pt}}}}
& & & & & & & 
& \question{\kn{ಮಗು ತನ್ನ ಪೂರ್ತಿ ಹೆಸರು ಹೇಳುತ್ತದೆಯೇ}} 
& \question{\kn{ಮಗು ತನ್ನ ವಯಸ್ಸು ಹೇಳುತ್ತದೆಯೇ}} 
& \question{\kn{ನಿನ್ನ ಅಂಗನವಾಡಿಯ ಹೆಸರೇನು?}} 
& \question{\kn{ನೀನು ಹುಡುಗನಾ?}} 
& \question{\kn{ನೀನು ಹುಡುಗಿಯಾ?}} 
& \question{\kn{ನಮ್ಮ ರಾಜ್ಯ ಯಾವುದು?}} 
& \question{\kn{ನಮ್ಮ ದೇಶ ಯಾವುದು?}} 
& \question{\kn{ಮಗು ಕುಟುಂಬದ ಸದಸ್ಯರ ದೂರವಾಣಿ ಸಂಖ್ಯೆ ಅಥವಾ ವಿಳಾಸ ಹೇಳುವುದು}} 
& \question{\kn{\makecell[b]{ಮಗು ಪೆನ್ಸಿಲ್ ಚೆನ್ನಾಗಿ ಹಿಡಿಯುವುದು \\(ಮಕ್ಕಳು ಚಟುವಟಿಕೆ ಹಾಳೆಯನ್ನು ಮಾಡುವಾಗ ಕಾರ್ಯಕರ್ತೆಗಮನಿಸುವುದು)}}}
& \question{\kn{\makecell[b]{ಮಗು ಒತ್ತಿ ಬರೆಯುವುದು \\(ಮಕ್ಕಳು ಚಟುವಟಿಕೆಹಾಳೆಯನ್ನು ಮಾಡುವಾಗ ಕಾರ್ಯಕರ್ತೆಗಮನಿಸುವುದು)}}}
& \question{\kn{ಆಕಾರದೊಳಗೆ ಬಣ್ಣ ಹಾಕುವುದು (ಚಟುವಟಿಕೆ ಹಾಳೆ 11)}}
& \question{\kn{ಅಂಗಿಯ ಗುಂಡಿ ತೆಗೆಯುವುದು}}
& \question{\kn{ಗುಂಡಿ ಹಾಕುವುದು}}
& \question{\kn{ಮಗು ಆಕಾರಗಳ ಮೇಲೆ ತಿದ್ದುವುದು (ಚಟುವಟಿಕೆ ಸಂಖ್ಯೆ 14) }} 
& \question{\kn{ಮಗು ಆಕಾರಗಳನ್ನು ನಕಲು ಮಾಡುವುದು (ಚಟುವಟಿಕೆ ಸಂಖ್ಯೆ 15)}} 
& \question{\kn{ಮಗು ಆಕ್ಷರಗಳನ್ನು ತಿದ್ದುವುದು (ಚಟುವಟಿಕೆ ಸಂಖ್ಯೆ 16)}} 
& \question{\kn{ಮಗು ಅಕ್ಷರಗಳನ್ನು ನಕಲು ಮಾಡುವುದು (ಚಟುವಟಿಕೆ ಸಂಖ್ಯೆ 17)}} 
& \question{\kn{ಮಗು ಅಕ್ಷರಗಳನ್ನು ತಿದ್ದುವುದು (ಚಟುವಟಿಕೆ ಸಂಖ್ಯೆ 18)}} 
& \question{\kn{ಮಗು ಅಕ್ಷರಗಳನ್ನು ನಕಲು ಮಾಡುವುದು (ಚಟುವಟಿಕೆ ಸಂಖ್ಯೆ 19)}} 
& \question{\kn{ಮಗು ಸಂಖ್ಯೆಗಳನ್ನು ತಿದ್ದುವುದು (ಚಟುವಟಿಕೆ ಸಂಖ್ಯೆ 20)}} 
& \question{\kn{ಮಗು ಸಂಖ್ಯೆಗಳನ್ನು ನಕಲು ಮಾಡುವುದು (ಚಟುವಟಿಕೆ ಸಂಖ್ಯೆ 21)}}
& \question{\kn{ಮಗು ತನ್ನ ಹೆಸರು ಬರೆಯುವುದು (ಗಮನಿಸಿ)}}
\\ \hline

& & & & & & & & \multicolumn{3}{|c|}{\kn{\makecell[b]{ಹೌದು=1,\\ ಇಲ್ಲ=0}}}
& \multicolumn{5}{|c|}{\kn{\makecell[b]{ಸರಿಯಾಗಿ ಹೇಳಿದರೆ =1,\\ ಹೇಳದಿದ್ದರೆ=0}}}
& \multicolumn{2}{|c|}{\kn{\makecell[b]{ಹೌದು=1,\\ ಇಲ್ಲ=0}}}
& \kn{\makecell[b]{4=1, \\4\\ಕ್ಕಿಂತ\\ಕಡಿಮೆ\\=0}}
& \multicolumn{2}{|c|}{\kn{\makecell[b]{ಹೌದು\\=1,\\ ಇಲ್ಲ\\=0}}}
& \kn{\makecell[b]{4=1, \\4\\ಕ್ಕಿಂತ\\ಕಡಿಮೆ\\	=0}}
& \multicolumn{7}{|c|}{\kn{3=1, 3 ಕ್ಕಿಂತ  ಕಡಿಮೆ=0}}
& \kn{\makecell[b]{ಹೆಸರು \\ ಬರೆದರೆ\\=1, \\ಬರೆಯ\\ದಿದ್ದರೆ=0}}
\\ \hline
CID-\kn{ಮಗುವಿನ ಗುರುತು} & \kn{ಮಗುವಿನ ಹೆಸರು} & \kn{ತಂದೆಯ ಹೆಸರು} & \kn{ತಾಯಿಯ ಹೆಸರು} & \kn{ಲಿಂಗ} & \kn{ಹುಟ್ಟಿದ ದಿನಾಂಕ} & \kn{ಎತ್ತರ} & \kn{ತೂಕ} 
& 1 & 2 & 3 & 4 & 5 & 6 & 7 & 8 & 9 & 10 & 11 & 12 & 13 & 14 & 15 & 16 & 17 & 18 & 19 & 20 & 21 & 22 \\ \hline
& & & & & & & &  & & & & & & & & & & & & & & & & & & & & & \\ \hline
& & & & & & & &  & & & & & & & & & & & & & & & & & & & & & \\ \hline
& & & & & & & &  & & & & & & & & & & & & & & & & & & & & & \\ \hline
& & & & & & & &  & & & & & & & & & & & & & & & & & & & & & \\ \hline
& & & & & & & &  & & & & & & & & & & & & & & & & & & & & & \\ \hline
& & & & & & & &  & & & & & & & & & & & & & & & & & & & & & \\ \hline
& & & & & & & &  & & & & & & & & & & & & & & & & & & & & & \\ \hline
& & & & & & & &  & & & & & & & & & & & & & & & & & & & & & \\ \hline
& & & & & & & &  & & & & & & & & & & & & & & & & & & & & & \\ \hline
& & & & & & & &  & & & & & & & & & & & & & & & & & & & & & \\ \hline

\end{tabular}

\pagebreak

\SetWatermarkText{Worker Copy}

\begin{tabular}{|p{2cm}|p{3cm}|p{2.8cm}|p{2.8cm}|p{1cm}|p{1cm}|l|l|l|l|l|l|l|l|l|l|l|l|l|l|l|}
\hline

& & & & & & & & \multicolumn{8}{|c|}{\kn{ಶಿಕ್ಷಣ ಪೂರ್ವ ಗಣಿತ}}
& \multicolumn{5}{|c|}{\kn{\makecell[b]{ಸಾಮಾಜಿಕ ಮತ್ತು \\ಭಾವನಾತ್ಮಕ ಬೆಳವಣಿಗೆ }}}
\\ \hline

& & & & & & & & \question{\kn{ಮಗು 1 ರಿಂದ 20 ರವರೆಗೆ ಸಂಖ್ಯೆ ಹೇಳುವುದು}}
& \question{\kn{\makecell[b]{ಅನುಕ್ರಮ ಜೋಡಣೆ ಮಾಡುವುದು \\ (ದೊಡ್ಡದರಿಂದ ಚಿಕ್ಕದು ಅಥವಾ ಚಿಕ್ಕದರಿಂದ ದೊಡ್ಡದು)}}}
& \question{\kn{ಯಾವುದರಲ್ಲಿ ಹೆಚ್ಚು ಮೀನುಗಳಿವೆ (ಚಟುವಟಿಕೆ ಸಂಖ್ಯೆ 25)}}
& \question{\kn{ಕಾರ್ಯಕರ್ತೆ ಹೇಳಿದ ಸಂಖ್ಯೆ ತೋರಿಸುವುದು (ಚಟುವಟಿಕೆ ಸಂಖ್ಯೆ 26)}}
& \question{\kn{1 ರಿಂದ 10 ರವರೆಗೆ ಸಂಖ್ಯೆ ಗುರುತಿಸುವುದು (ಚಟುವಟಿಕೆ ಸಂಖ್ಯೆ 27)}}
& \question{\kn{ಯಾವುದಾದರು 10 ವಸ್ತುಗಳನ್ನು ಎಣಿಸುವುದು}}
& \question{\kn{ಮಗು ಹೆಜ್ಜೆಯಿಂದ ಅಳತೆ ಮಾಡುವುದು}}
& \question{\kn{ಎಣಿಸಿ ಮೊತ್ತ ಬರೆಯುವುದು (ಚಟುವಟಿಕೆ ಸಂಖ್ಯೆ 30)}}
& \question{\kn{ಮಗು ಆಟಿಕೆಗಳನ್ನು ಸ್ನೇಹಿತರೊ೦ದಿಗೆ ಹಂಚಿಕೊಳ್ಳುವುದು}}
& \question{\kn{ಮಗು ಸರದಿಗಾಗಿ ಕಾಯುವುದು}}
& \question{\kn{ಮಗು ಸ್ನೇಹಿತರನ್ನು ಮಾಡಿಕೊಳ್ಳುವುದು}}
& \question{\kn{ಮಗು ಕೇಂದ್ರದ ವಸ್ತುಗಳನ್ನು ಜೋಪಾನ ಮಾಡುವುದು}}
& \question{\kn{ಮಗು ಕಾರ್ಯಕರ್ತೆಗೆ ಮತ್ತು ಸಹಾಯಕಿಗೆ ಸಹಾಯ ಮಾಡುವುದು}}
\\ \hline

& & & & & & & & \multicolumn{2}{|c|}{\kn{\makecell[b]{ಹೌದು=1, \\ ಇಲ್ಲ=0}}}
& \kn{\makecell[b]{ಸರಿ=1, \\ತಪ್ಪು=0}}
& \kn{\makecell[b]{ಕನಿಷ್ಠ 5 \\ಸಂಖ್ಯೆ \\ತೋರಿಸಿದರೆ=1, \\5 ಕ್ಕಿಂತ \\ಕಡಿಮೆ=0}}
& \kn{\makecell[b]{ಸರಿಯಾಗಿ \\ಗುರುತಿಸಿದರೆ\\=1, \\ಗುರುತಿಸದಿದ್ದರೆ\\=0}}
& \kn{\makecell[b]{ಹೌದು=1, \\ ಇಲ್ಲ=0}}
& \kn{\makecell[b]{ಸರಿಯಾಗಿ \\ಅಳತೆ \\ಮಾಡಿದರೆ\\=1, \\ ಮಾಡದಿದ್ದರೆ\\=0}}
& \kn{\makecell[b]{ಸರಿ=1,\\ ತಪ್ಪು=0}}
& \multicolumn{5}{|c|}{\kn{\makecell[b]{ಹೌದು=1, ಇಲ್ಲ=0 \\(ಕಾರ್ಯಕರ್ತೆ ಗಮನಿಸುವುದು)}}}
\\ \hline

CID-\kn{ಮಗುವಿನ ಗುರುತು} & \kn{ಮಗುವಿನ ಹೆಸರು} & \kn{ತಂದೆಯ ಹೆಸರು} & \kn{ತಾಯಿಯ ಹೆಸರು} & \kn{ಲಿಂಗ} & \kn{ಹುಟ್ಟಿದ ದಿನಾಂಕ} & \kn{ಎತ್ತರ} & \kn{ತೂಕ} & 23 & 24 & 25 & 26 & 27 & 28 & 29 & 30 & 31 & 32 & 33 & 34 & 35 \\ \hline
& & & & & & & &  & & & & & & & & & & & & \\ \hline
& & & & & & & &  & & & & & & & & & & & & \\ \hline
& & & & & & & &  & & & & & & & & & & & & \\ \hline
& & & & & & & &  & & & & & & & & & & & & \\ \hline
& & & & & & & &  & & & & & & & & & & & & \\ \hline
& & & & & & & &  & & & & & & & & & & & & \\ \hline
& & & & & & & &  & & & & & & & & & & & & \\ \hline
& & & & & & & &  & & & & & & & & & & & & \\ \hline
& & & & & & & &  & & & & & & & & & & & & \\ \hline
& & & & & & & &  & & & & & & & & & & & & \\ \hline

\end{tabular}

\pagebreak

\SetWatermarkText{Facilitator Copy}

\begin{tabular}{|p{2cm}|p{3cm}|p{2.8cm}|p{2.8cm}|p{1cm}|p{1cm}|l|l|l|l|l|l|l|l|l|l|l|l|l|l|}
\hline

& & & & & & & & \multicolumn{3}{|c|}{\kn{ಸ್ಥೂಲ ಚಲನೆ}}
& \multicolumn{9}{|c|}{\kn{ಭಾಷಾ ಬೆಳವಣಿಗೆ}}
\\ \hline

& & & & & & & & \question{\kn{\makecell[b]{ಚೆಂಡು ಎಸೆಯುವುದು (ಕನಿಷ್ಠ 3 ಬಾರಿ ಮಗು ನಿಮ್ಮ ನೇರಕ್ಕೆ ಅಥವಾ \\ನಿಮ್ಮ ಕಡೆಗೆ ಚೆಂಡನ್ನು ಎಸೆಯುವಂತೆ ಅವಕಾಶ ಕೊಡಬೇಕು. \\ಕನಿಷ್ಠ  ಒಂದು ಬಾರಿ ಮಗು ಚೆಂಡನ್ನು ನಿಮ್ಮ ಕಡೆಗೆ ಸರಿಯಾಗಿ ಎಸೆಯುವುದು)}}}
& \question{\kn{\makecell[b]{ಚೆಂಡು ಹಿಡಿಯುವುದು (ಕನಿಷ್ಠ 3 ಬಾರಿ ಮಗುವಿನ ಕಡೆ ಚೆಂಡನ್ನು ಎಸೆಯಿರಿ \\ ಕನಿಷ್ಠ ಒಂದು ಬಾರಿ ಮಗು ಚೆಂಡನ್ನು ಸರಿಯಾಗಿ ಹಿಡಿಯಬೇಕು)}}}
& \question{\kn{\makecell[b]{ಒಂಟಿ ಕಾಲಿನಲ್ಲಿ ಕುಂಟುವುದು (ಗುಂಪು ಚಟುವಟಿಕೆ) \\(ಕನಿಷ್ಟ 5 ಹೆಜ್ಜೆ)}}}
& \question{\kn{ಮಗು ಅಭಿನಯದೊಂದಿಗೆ  ಶಿಶುಗೀತೆ ಹೇಳುವುದು}} 
& \question{\kn{\makecell[b]{ಮಗು ಪ್ರಶ್ನೆಗೆ ವಾಕ್ಯದಲ್ಲಿ ಉತ್ತರಿಸುವುದು.\\ ಕನಿಷ್ಠ 3 ಪ್ರಶ್ನೆ (ವಾರದ ವಿಷಯಕ್ಕೆ ಸಂಬಂದಿಸಿದಂತೆ)}}} 
& \question{\kn{ಮಗು 3 ಸೂಚನೆಗಳನ್ನು ಸರಿಯಾದ ಕ್ರಮದಲ್ಲಿ ಪಾಲಿಸುತ್ತಾಳೆ/ನೆ}} 
& \question{\kn{\makecell[b]{5 ವಸ್ತುಗಳನ್ನು ಇಂಗ್ಲೀಷ್ ನಲ್ಲಿ ಹೆಸರಿಸುತ್ತಾಳೆ/ನೆ  \\ (ವಿಷಯಕ್ಕೆ ಸಂಭಂದಿಸಿದಂತೆ )}}} 
& \question{\kn{6 ಬಣ್ಣಗಳನ್ನು ಹೆಸರಿಸುತ್ತಾರೆ (ಕೆಂಪು, ನೀಲಿ, ಹಳದಿ, ಕಪ್ಪು, ಬಿಳಿ, ಹಸಿರು)}} 
& \question{\kn{\makecell[b]{4 ಆಕಾರಗಳನ್ನು ಹೆಸರಿಸುವುದು (ತ್ರಿಭುಜ, ವೃತ್ತ, ಚೌಕ, ಆಯತ)\\ (ಚಟುವಟಿಕೆ ಹಾಳೆ-11)}}} 
& \question{\makecell[b]{\kn{2 ಪ್ರಶ್ನೆಗಳಿಗೆ ಮಗು ಇಂಗ್ಲೀಷ್ ನಲ್ಲಿ ಉತ್ತರಿಸುವುದು } \\ 1. what is your name? \\ 2. How old are you?}}
& \question{\kn{ಮಗು ಎಡ-ಬಲ ತೋರಿಸುವುದು}} 
& \question{\kn{ಮಗು ಹಿಂದೆ ಮುಂದೆ ತೋರಿಸುವುದು}} 
%& \question{\kn{ಕಥೆ ಹೇಳುವುದು (೪-೫ ಸಾಲಿನ ಕಥೆ)}} 
%& \question{\kn{ವಾರದ ಹೆಸರು ಹೇಳುವುದು (ಅನುಕ್ರಮದಲ್ಲಿ ಹೇಳುವುದು)}} 
%& \question{\kn{ತಿಂಗಳ ಹೆಸರುಗಳನ್ನು ಹೇಳುವುದು (ಅನುಕ್ರಮದಲ್ಲಿ ಹೇಳುವುದು)}}
\\ \hline

& & & & & & & & \kn{\makecell[b]{ಸರಿಯಾಗಿ \\ ಎಸೆದರೆ\\=1, \\ ಎಸೆಯದಿದ್ದರೆ\\=0}}
& \kn{\makecell[b]{ಸರಿಯಾಗಿ \\ ಹಿಡಿದರೆ\\=1, \\ ಹಿಡಿಯದಿದ್ದರೆ\\=0}}
& \kn{\makecell[b]{ಒಂಟಿ \\ ಕಾಲಿನಲ್ಲಿ \\ ಕುಂಟಿದರೆ\\=1, \\ ಕುಂಟದಿದ್ದರೆ\\=0}}
& \kn{\makecell[b]{ಅಭಿನಯ\\ದೊಂದಿಗೆ  \\ ಹೇಳಿದರೆ=1, \\ ಶಿಶುಗೀತೆ\\ ಮಾತ್ರ  \\ ಹೇಳಿದರೆ=0}}
& \multicolumn{2}{|c|}{\kn{\makecell[b]{3=1, \\ 3 ಕ್ಕಿಂತ \\ ಕಡಿಮೆ=0}}}
& \kn{\makecell[b]{5=1, \\ 5 ಕ್ಕಿಂತ \\ ಕಡಿಮೆ\\=0}} 
& \kn{\makecell[b]{6=1, \\ 6 ಕ್ಕಿಂತ \\ ಕಡಿಮೆ\\=0}} 
& \kn{\makecell[b]{4 \\ಆಕಾರಗಳು \\= 1, \\ 4 ಕ್ಕಿಂತ \\ಕಡಿಮೆ \\= 0}} 
& \kn{\makecell[b]{2=1, \\ 2 ಕ್ಕಿಂತ \\ ಕಡಿಮೆ=0}} 
& \multicolumn{2}{|c|}{\kn{\makecell[b]{ತೋರಿಸಿದರೆ \\=1, \\ ತೋರಿಸದಿದ್ದರೆ\\=0}}}
%& \kn{\makecell[b]{ಪೂರ್ತಿ\\ ಕಥೆ \\ಹೇಳಿದರೆ=1, \\ ಹೇಳದಿದ್ದರೆ=0}} 
%& \multicolumn{2}{|c|}{\kn{\makecell[b]{ಪೂರ್ತಿ\\ ಹೇಳಿದರೆ=1, \\ ಹೇಳದಿದ್ದರೆ=0}}} 
\\ \hline

CID-\kn{ಮಗುವಿನ ಗುರುತು} & \kn{ಮಗುವಿನ ಹೆಸರು} & \kn{ತಂದೆಯ ಹೆಸರು} & \kn{ತಾಯಿಯ ಹೆಸರು} & \kn{ಲಿಂಗ} & \kn{ಹುಟ್ಟಿದ ದಿನಾಂಕ} & \kn{ಎತ್ತರ} & \kn{ತೂಕ} & 36 & 37 & 38 & 39 & 40 & 41 & 42 & 43 & 44 & 45 & 46 & 47 \\ \hline
& & & & & & & &  & & & & & & & & & & & \\ \hline
& & & & & & & &  & & & & & & & & & & & \\ \hline
& & & & & & & &  & & & & & & & & & & & \\ \hline
& & & & & & & &  & & & & & & & & & & & \\ \hline
& & & & & & & &  & & & & & & & & & & & \\ \hline
& & & & & & & &  & & & & & & & & & & & \\ \hline
& & & & & & & &  & & & & & & & & & & & \\ \hline
& & & & & & & &  & & & & & & & & & & & \\ \hline
& & & & & & & &  & & & & & & & & & & & \\ \hline
& & & & & & & &  & & & & & & & & & & & \\ \hline

\end{tabular}

\pagebreak

\SetWatermarkText{Facilitator Copy}

\begin{tabular}{|p{2cm}|p{3cm}|p{2.6cm}|p{2.6cm}|p{0.7cm}|p{1cm}|l|l|l|l|l|l|l|l|l|l|l|l|l|l|l|l|l|}
\hline

& & & & & & & & \multicolumn{3}{|c|}{\kn{ಭಾಷಾ ಬೆಳವಣಿಗೆ}}
& \multicolumn{3}{|c|}{\kn{ಗ್ರಹಣ ಶಕ್ತಿ ಬೆಳವಣಿಗೆ}}
& \multicolumn{4}{|c|}{\kn{\makecell[b]{ಶಿಕ್ಷಣ ಪೂರ್ವ ಓದುವುದು \\(ಇಂಗ್ಲೀಷ್ )}}}
& \multicolumn{5}{|c|}{\kn{ಶಿಕ್ಷಣ ಪೂರ್ವ ಓದುವುದು (ಕನ್ನಡ)}}
\\ \hline

& & & & & & & %& \question{\makecell[b]{\kn{2 ಪ್ರಶ್ನೆಗಳಿಗೆ ಮಗು ಇಂಗ್ಲೀಷ್ ನಲ್ಲಿ ಉತ್ತರಿಸುವುದು } \\ 1. what is your name? \\ 2. How old are you?}}
%& \question{\kn{ಮಗು ಎಡ - ಬಲ ತೋರಿಸುವುದು}} 
%& \question{\kn{ಮಗು ಹಿಂದೆ ಮುಂದೆ ತೋರಿಸುವುದು}} 
& \question{\kn{ಕಥೆ ಹೇಳುವುದು (4-5 ಸಾಲಿನ ಕಥೆ)}} 
& \question{\kn{ವಾರದ ಹೆಸರು ಹೇಳುವುದು (ಅನುಕ್ರಮದಲ್ಲಿ ಹೇಳುವುದು)}} 
& \question{\kn{ತಿಂಗಳ ಹೆಸರುಗಳನ್ನು ಹೇಳುವುದು (ಅನುಕ್ರಮದಲ್ಲಿ ಹೇಳುವುದು)}}
& \question{\kn{ಮಗು ವಸ್ತುಗಳನ್ನು ಬೇರ್ಪಡಿಸುವುದು (ಎಲೆ, ಕಲ್ಲು, ಬೀಜ)}} 
& \question{\kn{5 ವಸ್ತುಗಳ ಪುನರುಚ್ಛಾರಣೆ }}
& \question{\kn{\makecell[b]{ನನ್ನ ದಿನ ಮಿಂಚು ಪಟ್ಟಿಗಳನ್ನು ಅನುಕ್ರಮದಲ್ಲಿ ಜೋಡಿಸುವುದು \\ (ಇಲಾಖೆಯ ಕಲಿಕಾ ಸಾಮಗ್ರಿ ಉಪಯೋಗಿಸುವುದು)}}}
& \question{A \kn{ಯಿಂದ} Z \kn{ವರೆಗೆ ಹೇಳುವುದು }}
& \question{A \kn{ಯಿಂದ} Z \kn{ವರೆಗೆ ಗುರುತಿಸುವುದು (ಮಿಂಚು ಪಟ್ಟಿ ಉಪಯೋಗಿಸುವುದು) }}
& \question{\kn{ಅಕ್ಷರದ ಶಬ್ಧಗಳನ್ನು ಸರಿಯಾಗಿ ಉಚ್ಛರಿಸುವುದು (ಮಿಂಚು ಪಟ್ಟಿ ಉಪಯೋಗಿಸುವುದು)}}
& \question{\kn{ಪದದ ಪ್ರಾರಂಭದ ಅಕ್ಷರವನ್ನು ಗುರುತಿಸುವುದು (ಮಿಂಚು ಪಟ್ಟಿ ಉಪಯೋಗಿಸುವುದು)}}
& \question{\kn{ಅ ಇಂದ ಳ  ವರೆಗೆ ಹೇಳುತ್ತಾಳೆ/ನೆ  (ಮಿಂಚು ಪಟ್ಟಿ ಉಪಯೋಗಿಸಿ)}}
& \question{\kn{\makecell[b]{ಅ ಇಂದ ಳ ವರೆಗೆ ಗುರುತಿಸುತ್ತಾಳೆ/ನೆ (ಮಿಂಚು ಪಟ್ಟಿ ಉಪಯೋಗಿಸಿ) }}}
& \question{\kn{ಪ್ರಾರಂಭದ ಅಕ್ಷರ ಗುರುತಿಸುತ್ತಾಳೆ/ನೆ  (ಮಿಂಚು ಪಟ್ಟಿ ಉಪಯೋಗಿಸುವುದು)}}
& \question{\kn{\makecell[b]{ಅಕ್ಷರದಿಂದ ಪ್ರಾರಂಭವಾಗುವ ಪದ ಹೇಳುತ್ತಾಳೆ/ನೆ  \\ (ಉದಾ: ಅ-ಅರಸ , ಉ-ಉದಯ ಇತ್ಯಾದಿ)}}}
& \question{\kn{\makecell[b]{3 ಸರಳ ಪದ ಓದುವುದು (ಮಿಂಚು ಪಟ್ಟಿ ಉಪಯೋಗಿಸುವುದು) \\ (ಅರಸ, ಉದಯ, ಔಷದ)}}}
\\ \hline

& & & & & & & %& \kn{\makecell[b]{2=1, \\ 2 ಕ್ಕಿಂತ \\ ಕಡಿಮೆ=0}} 
%& \multicolumn{2}{|c|}{\kn{\makecell[b]{ತೋರಿಸಿದರೆ \\=1, \\ ತೋರಿಸದಿದ್ದರೆ\\=0}}}
& \multicolumn{3}{|c|}{\kn{\makecell[b]{ಪೂರ್ತಿ\\ ಹೇಳಿದರೆ\\=1, \\ ಹೇಳದಿದ್ದರೆ\\=0}}}
& \kn{\makecell[b]{ಬೇರ್ಪಡಿ\\ಸಿದರೆ=1, \\ ಬೇರ್ಪಡಿಸ\\ದಿದ್ದರೆ=0}}
& \kn{\makecell[b]{5 ವಸ್ತುಗಳು\\=1, \\ 5 ಕ್ಕಿಂತ \\ಕಡಿಮೆ=0}}
& \kn{\makecell[b]{ಸರಿಯಾಗಿ \\ ಜೋಡಿಸಿ\\ದರೆ=1, \\ ಜೋಡಿಸದಿ\\ದ್ದರೆ=0}}
& \kn{\makecell[b]{ಕನಿಷ್ಠ 3 \\ ತಪ್ಪು ಉತ್ತರ \\ಇದ್ದರೆ=1, \\3ಕ್ಕಿಂತ \\ಜಾಸ್ತಿ ತಪ್ಪು \\ಇದ್ದರೆ=0}} 
& \multicolumn{3}{|c|}{\kn{\makecell[b]{10=1,\\ 10ಕ್ಕಿಂತ \\ಕಡಿಮೆ=0}}} 
& \kn{\makecell[b]{ಹೌದು=1,\\ ಇಲ್ಲ=0}}
& \multicolumn{2}{|c|}{\kn{\makecell[b]{15 ಅಕ್ಷರ=1, \\15 ಕ್ಕಿಂತ \\ಕಡಿಮೆ=0}}}
& \kn{\makecell[b]{ಕನಿಷ್ಠ \\ 1 ಪದ  \\ಹೇಳಿದರೆ\\=1, \\ಹೇಳದಿ\\ದ್ದರೆ=0}} 
& \kn{\makecell[b]{3=1, \\3 ಕ್ಕಿಂತ \\ಕಡಿಮೆ\\=0}}
\\ \hline

CID-\kn{ಮಗುವಿನ ಗುರುತು} & \kn{ಮಗುವಿನ ಹೆಸರು} & \kn{ತಂದೆಯ ಹೆಸರು} & \kn{ತಾಯಿಯ ಹೆಸರು} & \kn{ಲಿಂಗ} & \kn{ಹುಟ್ಟಿದ ದಿನಾಂಕ} & \kn{ಎತ್ತರ} & \kn{ತೂಕ} & 48 & 49 & 50 & 51 & 52 & 53 & 54 & 55 & 56 & 57 & 58 & 59 & 60 & 61 & 62 \\ \hline
& & & & & & & &  & & & & & & & & & & & & & & \\ \hline
& & & & & & & &  & & & & & & & & & & & & & & \\ \hline
& & & & & & & &  & & & & & & & & & & & & & & \\ \hline
& & & & & & & &  & & & & & & & & & & & & & & \\ \hline
& & & & & & & &  & & & & & & & & & & & & & & \\ \hline
& & & & & & & &  & & & & & & & & & & & & & & \\ \hline
& & & & & & & &  & & & & & & & & & & & & & & \\ \hline
& & & & & & & &  & & & & & & & & & & & & & & \\ \hline
& & & & & & & &  & & & & & & & & & & & & & & \\ \hline
& & & & & & & &  & & & & & & & & & & & & & & \\ \hline

\end{tabular}

\end{document}
