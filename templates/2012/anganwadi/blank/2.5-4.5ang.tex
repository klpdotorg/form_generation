\documentclass[12pt]{article}
%%\usepackage[normalmargins]{savetrees}
\usepackage[a3paper,top=2cm,left=0.5cm,right=0.5cm,bottom=2cm,landscape]{geometry}
\usepackage{rotating,makecell}
\usepackage{fontenc}
\usepackage{draftwatermark}
%% \usepackage{xunicode}
\usepackage{xltxtra}
\usepackage{graphicx}
\usepackage{lastpage}
\usepackage{fp}
\usepackage{fancyhdr}
\pagestyle{fancy}

\def\heading{2011-12 Anganwadi assessment}
%% Headers and footers
\lhead{\includegraphics[scale=0.15]{/home/shivangi/src/assessments/form_generation/logo.jpg}}
\chead{2years 6months – 4years 6months age group \\ID: \rule{2in}{0.5pt} Name: \rule{2in}{0.5pt}  Type: \rule{2in}{0.5pt}}
\rhead{Project: \rule{2in}{0.5pt} Circle: \rule{2in}{0.5pt}\\ Worker: \rule{2in}{0.5pt}}
\cfoot{\small{Akshara -- Anganwadi Assessment -- \today © Akshara Foundation}}
\rfoot{\small{\thepage of \pageref{LastPage}}}


%\nomakegapedcells
%\renewcommand\theadset{\renewcommand\arraytretch{1}%
%    \setlength\extrarowheight{0cm}}
%\renewcommand\theadgape{0cm}
%\renewcommand\cellgape{}
%\setcellgapes[b]{-0.5cm}

\setlength{\headheight}{15pt}
\renewcommand{\headrulewidth}{0.4pt}
\renewcommand{\footrulewidth}{0.4pt}

\newcommand{\qsetnum}{%
\FPsub{\qnum}{\thepage}{1}
\FPdiv{\qnum}{\thepage}{4}%
\FPtrunc{\tqnum}{\qnum}{0}%
\FPsub{\qnum}{\qnum}{\tqnum}%
\FPmul{\qnum}{\qnum}{4}%
\FPtrunc{\qnum}{\qnum}{0}%
\FPifeq{\qnum}{0}
   \FPset{\qnum}{4}%
\fi
\FPprint{\qnum}
}%

\newcommand{\kn}[1]{%Kannada text
{\fontspec[Script=Kannada]{Kedage}%
#1
}}

%\newcommand{\grading}[1]{\begin{sideways}\textbf{#1}\end{sideways}}
%\newcommand{\instruct}[1]{\begin{sideways}\textbf{#1}\end{sideways}}
\newcommand{\question}[1]{\begin{sideways}#1\end{sideways}}
\renewcommand{\arraystretch}{1.4}

\title{\heading}
\begin{document}

%\fontspec[Script=Kannada]{Kedage}
\SetWatermarkText{1}

\begin{tabular}{|l|p{3cm}|p{3cm}|p{3cm}|p{1.5cm}|p{1.5cm}|l|l|l|l|l|l|l|l|l|l|l|l|l|l|}
\hline
& & & & & & & &\multicolumn{3}{|c|}{\kn{ಸಾಮಾನ್ಯ ಅರಿವು}} & \multicolumn{4}{|c|}{\kn{ಸ್ಥೂಲ ಚಲನೆಯ ಕೌಶಲಗಳು}} & \multicolumn{5}{|c|}{\kn{ಸೂಕ್ಷ್ಮ ಚಲನೆಯ ಕೌಶಲಗಳು}} \\
\cline{16-20}
& & & & & & & & \multicolumn{3}{|c|}{} & \multicolumn{4}{|c|}{} & \multicolumn{4}{|c|}{\kn{ಸ್ವಸಹಾಯ ಕೌಶಲಗಳು}} & \multicolumn{1}{|c|}{\kn{\makecell[b]{ಕೈ ಚಳಕ ಬಳಕೆಯ\\ ಕೌಶಲಗಳು}}} \\ \hline
& & & & & & &  &\question{\kn{ಮಗು ತನ್ನ ಹೆಸರು ಹೇಳುತ್ತದೆಯೇ}} & \question{\kn{ತಾಯಿಯ ಹೆಸರು ಹೇಳುತ್ತದೆಯೇ}} & \question{\kn{ತಂದೆಯ ಹೆಸರು ಹೇಳುತ್ತದೆಯೇ}} & \question{\kn{\makecell[b]{ಚೆಂಡನ್ನು ಹಿಡಿಯುವುದು (ಮೂರು ಬಾರಿ ಮಗುವಿನ ಕಡೆ ಚೆಂಡನ್ನು ಎಸೆಯಿರಿ, \\ಕನಿಷ್ಠ ಒಂದು ಬಾರಿ ಮಗು ಚೆಂಡನ್ನು  ಸರಿಯಾಗಿ ಹಿಡಿಯ ಬೇಕು)  \\(ಗುಂಪು ಚಟುವಟಿಕೆ)}}} & \question{\kn{\makecell[b]{ಚೆಂಡನ್ನು ಎಸೆಯುವುದು (ಮೂರು ಬಾರಿ ಮಗು ನಿಮ್ಮ ನೇರಕ್ಕೆ ಅಥವಾ ನಿಮ್ಮ \\ಕಡೆಗೆ ಚೆಂಡನ್ನು ಎಸೆಯ ಬೇಕು, ಕನಿಷ್ಠ ಒಂದು ಬಾರಿ ಮಗು ಚೆಂಡನ್ನು ನಿಮ್ಮ \\ಕಡೆಗೆ  ಸರಿಯಾಗಿ ಎಸೆಯ ಬೇಕು) (ಗುಂಪು ಚಟುವಟಿಕೆ)}}} & \question{\kn{\makecell[b]{ಜಿಗಿಯುವುದು (ಮೂರು ಬಾರಿ ಜಿಗಿಯಲಿ, ಕನಿಷ್ಠ ಒಂದು ಬಾರಿ ಸರಿಯಾಗಿ \\ಜಿಗಿಯ ಬೇಕು)  (ಗುಂಪು ಚಟುವಟಿಕೆ)}}} & \question{\kn{ನೇರವಾದ ಗೆರೆಯ ಮೇಲೆ ನಡೆಯುವುದು  (ಗುಂಪು ಚಟುವಟಿಕೆ)}} & \question{\kn{ಅಂಗಿಯ ಗುಂಡಿಗಳನ್ನು ಹಾಕುವುದು}} & \question{\kn{ಅಂಗಿಯ ಗುಂಡಿಗಳನ್ನು ಬಿಚ್ಚುವುದು}} & \question{\kn{ನೀರನ್ನು ಸುರಿಯುವುದು}} & \question{\kn{\makecell[b]{ಮಗು ಯಾರ ಸಹಾಯವಿಲ್ಲದೆ ತನ್ನ ಊಟ ತಾನೇ ತಿನ್ನುತ್ತಾಳೆ/ನೆ. \\(ಅಂಗನವಾಡಿ ಕಾರ್ಯಕರ್ತೆಯ ಗಮನಿಸುವಿಕೆ)}}} & \question{\kn{ಗೋಪುರ ನಿರ್ಮಾಣ (ಐದು ಘನಾಕೃತಿಗಳು)}}  \\ \hline
& & & & & & & &\multicolumn{3}{|c|}{\kn{\makecell[b]{ಹೌದು=1,\\ ಇಲ್ಲ=0}}}&\kn{\makecell[b]{ಹಿಡಿದರೆ =1,\\ ಹಿಡಿಯದಿದ್ದರೆ \\= 0}}&\kn{\makecell[b]{ಎಸೆದರೆ =1,\\ ಎಸೆಯದಿದ್ದರೆ \\=0}}&\kn{\makecell[b]{ಜಿಗಿದರೆ =1,\\ ಜಿಗಿಯದಿದ್ದರೆ \\=0}}&\multicolumn{3}{|c|}{\kn{\makecell[b]{ಹೌದು=1, \\ಇಲ್ಲ=0}}}&\kn{\makecell[b]{ಚೆಲ್ಲದೆ\\ ಸುರಿದರೆ \\=1,\\ ಚೆಲ್ಲಿದರೆ = 0}}&\multicolumn{2}{|c|}{\kn{\makecell[b]{ಹೌದು=1,\\ ಇಲ್ಲ=0}}} \\ \hline
\makecell[b]{CID-\\ \kn{ಮಗುವಿನ ಗುರುತು}} & \kn{ಮಗುವಿನ ಹೆಸರು} & \kn{ತಂದೆಯ ಹೆಸರು} & \kn{ತಾಯಿಯ ಹೆಸರು} & \kn{ಲಿಂಗ} & \kn{ಹುಟ್ಟಿದ ದಿನಾಂಕ} & \kn{ಎತ್ತರ} & \kn{ತೂಕ}  & 1 & 2 & 3 & 4 & 5 & 6 & 7 & 8 & 9 & 10 & 11 &12 \\ \hline
 &  &  &  &  &  & & & & & & & & & & & & & &  \\ \hline
 &  &  &  &  &  & & & & & & & & & & & & & &  \\ \hline
 &  &  &  &  &  & & & & & & & & & & & & & &  \\ \hline
 &  &  &  &  &  & & & & & & & & & & & & & &  \\ \hline
 &  &  &  &  &  & & & & & & & & & & & & & &  \\ \hline
 &  &  &  &  &  & & & & & & & & & & & & & &  \\ \hline
 &  &  &  &  &  & & & & & & & & & & & & & &  \\ \hline
 &  &  &  &  &  & & & & & & & & & & & & & &  \\ \hline
 &  &  &  &  &  & & & & & & & & & & & & & &  \\ \hline
 &  &  &  &  &  & & & & & & & & & & & & & &  \\ \hline

\end{tabular}

\pagebreak

\SetWatermarkText{2}

\begin{tabular}{|l|p{3cm}|p{3cm}|p{3cm}|p{1.5cm}|p{1.5cm}|l|l|l|l|l|l|l|l|l|l|l|l|}
\hline
 & & & & & & & & \multicolumn{4}{|c|}{\kn{ಭಾಷಾ ಬೆಳವಣಿಗೆ}} & \multicolumn{4}{|c|}{\kn{ಸಾಮಾಜಿಕ ಭಾವನಾತ್ಮಕ ಬೆಳವಣಿಗೆ}} & \multicolumn{2}{|c|}{\kn{ಶಿಕ್ಷಣ ಪೂರ್ವ ಗಣಿತ}}\\ \hline
& & & & & & & &\question{\kn{ಶಿಶು ಗೀತೆಯನ್ನು ಹೇಳುವುದು (2 ರಿಂದ 4 ಸಾಲು)}} & \question{\kn{\makecell[b]{ಕ್ರಿಯೆಗಳನ್ನು ಗುರುತಿಸುವುದು (ಯಾವುದೇ 3 ಕ್ರಿಯೆಗಳನ್ನು ಮಾಡಿ ತೋರಿಸಿ, \\ಮಗು ಕ್ರಿಯೆಗಳನ್ನು ಸರಿಯಾಗಿ ಗುರುತಿಸಬೇಕು)}}}& \question{\kn{ದೇಹದ ಅಂಗಗಳನ್ನು ಗುರುತಿಸುವುದು (ತಲೆ, ಕಣ್ಣು, ಕೈ, ಕಾಲು, ಹೊಟ್ಟೆ) }}& \question{\kn{ಮಗು ಮೇಲೆ ಕೆಳಗೆ ತೋರಿಸುವುದು (ಮೇಲೆ-ಕೆಳಗೆ ಸರಿಯಾಗಿ ತೋರಿಸುವುದು)}}& \question{\kn{ಮಗು ಆಟಿಕೆಗಳನ್ನು ಸ್ನೇಹಿತರೊಂದಿಗೆ ಹಂಚಿಕೊಳ್ಳುವುದು}}& \question{\kn{ಮಗು ತನ್ನ ಸರದಿಗಾಗಿ ಕಾಯುವುದು}}& \question{\kn{ಮಗು ಭಾವನೆಗಳನ್ನು ವ್ಯಕ್ತಪಡಿಸುವುದು }}& \question{\kn{ಮಗು ಸ್ನೇಹಿತರನ್ನು ಮಾಡಿಕೊಳ್ಳುವುದು}}& \question{\kn{ಮಗು 1 ರಿಂದ 10 ರವರೆಗಿನ ಸಂಖ್ಯೆ ಹೇಳುವುದು}}& \question{\kn{ಮಗು ಹೆಚ್ಚು ಕಡಿಮೆ ಹೇಳುವುದು (ಕಲಿಕಾ ಸಾಮಗ್ರಿ ಬಳಸಿ)}}  \\ \hline
& & & & & & & & \kn{\makecell[b]{ಹೌದು\\=1,\\ ಇಲ್ಲ=0}} &\kn{\makecell[b]{3 = 1, \\3 \\ಕ್ಕಿಂತ ಕಡಿಮೆ\\ = 0}}&\kn{\makecell[b]{5=1,\\  5 \\ಕ್ಕಿಂತ ಕಡಿಮೆ\\ = 0}}&\kn{\makecell[b]{ತೋರಿಸಿದರೆ \\=1,\\ ತೋರಿಸದಿದ್ದರೆ\\ = 0}}&\multicolumn{4}{|c|}{\kn{\makecell[b]{(ಅಂಗನವಾಡಿ ಕಾರ್ಯಕರ್ತೆಯ\\ ಗಮನಿಸುವಿಕೆ)\\   ಹೌದು=1, ಇಲ್ಲ=0}}}&\kn{\makecell[b]{10=1,   \\ 10 ಕ್ಕಿಂತ ಕಡಿಮೆ\\ =0}} & \kn{\makecell[b]{ಹೌದು=1,\\ಇಲ್ಲ=0}}\\ \hline
\makecell[b]{CID-\\ \kn{ಮಗುವಿನ ಗುರುತು}} & \kn{ಮಗುವಿನ ಹೆಸರು} & \kn{ತಂದೆಯ ಹೆಸರು} & \kn{ತಾಯಿಯ ಹೆಸರು} & \kn{ಲಿಂಗ} & \kn{ಹುಟ್ಟಿದ ದಿನಾಂಕ} & \kn{ಎತ್ತರ} & \kn{ತೂಕ}& 13 & 14 & 15 & 16 & 17 & 18 & 19 & 20 & 21 & 22   \\ \hline
& &  &  &  &  &  & & & & & & & & & & &  \\ \hline
& &  &  &  &  &  & & & & & & & & & & &  \\ \hline
& &  &  &  &  &  & & & & & & & & & & &  \\ \hline
& &  &  &  &  &  & & & & & & & & & & &  \\ \hline
& &  &  &  &  &  & & & & & & & & & & &  \\ \hline
& &  &  &  &  &  & & & & & & & & & & &  \\ \hline
& &  &  &  &  &  & & & & & & & & & & &  \\ \hline
& &  &  &  &  &  & & & & & & & & & & &  \\ \hline
& &  &  &  &  &  & & & & & & & & & & &  \\ \hline
& &  &  &  &  &  & & & & & & & & & & &  \\ \hline 

\end{tabular}
\end{document}
